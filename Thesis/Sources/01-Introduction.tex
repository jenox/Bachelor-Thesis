\chapter{Introduction}
\label{chap:introduction}

In graph theory, a graph is an object consisting of a set of elements, called \emph{vertices}, and their pairwise relations, called \emph{edges}, that are expressed as (unordered) pairs of vertices. Graphs are used heavily for modeling routing problems, representing social networks, chip design, and many other fields.

Visualizing graphs is a fundamental aspect in the field of graph drawing and information visualization. A graph's vertices are typically drawn as small circles, with its edges being drawn as curves between their endpoints. When drawing graphs, one of the main goals is making them easy to grasp.

A popular family of algorithms for drawing graphs are so-called force-directed algorithms. In a force-directed algorithm, the graph is regarded as a particle system in which the vertices are particles, and several forces are acting on said particles. One defines these forces such that they act to bring the system into a stable equilibrium position in which its implicitly defined potential energy is at a local minimum and the resulting drawing has some desired features. Typically, one wants adjacent vertices to be close together, with non-adjacent vertices being further apart from each other. The resulting drawings of graphs are generally visually appealing and easy to grasp \cite{Kobourov}.

\section{Motivation}
\label{sect:motivation}

In traditional force-directed algorithms, the only features desired in equilibrium are that adjacent vertices are close together and that non-adjacent vertices are further apart from each other. Depending on what one wants to achieve when drawing a graph, one may have additional features \emdash or different features altogether \emdash that make a good drawing. The challenge then is to find appropriate restoring forces that act to put the system into a state of equilibrium that exhibits said features. Some features in graph drawings may be so indispensable that they pose hard constraints that have to be satisfied for the drawing to be acceptable. These constraints effectively restrict the vertices' movement, making traditional force-directed algorithms (in which vertices can move freely in both dimensions) inapplicable for finding a local energy minimum.

Schulz \cite{Schulz} proposed that graph drawings with low visual complexity, \ie{} drawings with few geometric entities, are easy to perceive by the viewer. Instead of drawing vertices and edges as their own geometric entity, Schulz aimed to draw multiple incident edges using a single geometric entity; namely, a circular arc. Considering there may not exist a circular arc through four or more arbitrarily placed vertices, this form of hard constraint reduces the number of degrees of freedom in the system.

This thesis serves to discuss an approach to minimizing the potential energy in systems whose constraints reduce its number of degrees of freedom. We will also demonstrate how this approach can be applied to creating drawings of graphs in which circular arcs are used to draw multiple incident edges.

\section{Related Work}
\label{sect:related-work}

In his research, Schulz dealt with the theoretical lower and upper bounds of the number of required entities and provided an algorithm satisfying said bounds only for very specific classes of graphs \cite{Schulz}. Similar research has been performed for drawing multiple incident edges using straight line segments \cite{Dujmovic, Durocher, Igamberdiev}.

Force-directed algorithms have already been used for drawing graphs with additional features desired in equilibrium.

Chernobelskiy \etal{} \cite{Chernobelskiy}, for example, studied so-called Lombardi drawings and used a force-directed approach to optimize angular resolution by introducing a new set of forces. Sugiyama and Misue \cite{Sugiyama} studied force-directed algorithms in which forces exerted by a magnetic field are used to align edges in a certain direction. In both cases, the additional desired feature is dispensable, and therefore only poses a soft constraint affecting the quality of a drawing. Although nice to have, valid drawings can be produced even if these constraints are violated.

Bertault \cite{Bertault} designed a force-directed algorithm called PrEd that preserves edge crossing from an initial layout. The additional desired feature here is indispensable and poses a hard constraint that has to be satisfied for the resulting drawing to be acceptable. Even though arbitrary displacements of the graph's vertices might render some of its drawings invalid, the additional constraint does not affect the number of degrees of freedom in the system.

