\thispagestyle{plain}

\begin{addmargin}{0.5cm}

\centerline{\textbf{Abstract}}

Over time, physical systems converge to stable equilibrium states in which their potential energy is at a local minimum. For systems that are not subject to constraints, force-directed algorithms can be used to find equilibrium positions. However, many systems are subject to constraints, restricting the movement of particles in the system. We propose an alternative method to minimize potential energy and thereby find stable equilibrium positions, that can be used for constrained systems.

We then demonstrate how this can be applied in the field of graph drawing. Here the constraints appear in the form of multiple incident edges being drawn using a single circular arc.

\vskip 2cm

\centerline{\textbf{Deutsche Zusammenfassung}}

\selectlanguage{ngerman}
Physikalische Systeme konvergieren mit der Zeit gegen einen stabilen Gleichgewichtszustand, in dem die potentielle Energie des Systems ein lokales Minimum annimmt. In Systemen, die keinen Zwangsbedingungen unterworfen sind, können sogenannte \emph{Force-Directed Algorithmen} verwendet werden, um Gleichgewichtszustände zu finden. Viele Systeme sind jedoch Zwangsbedingungen unterworfen, die die Bewegungsfreiheit der Partikel im System einschränken. Wir stellen eine alternative Methode zur Minimierung der potentiellen Energie vor, die auch für solche Systeme benutzt werden kann.

Wir zeigen anschließend, wie diese Methode im Bereich des Graphenzeichnens angewendet werden kann. Die Zwangsbedingungen bestehen hierbei darin, dass mehrere inzidente Kanten durch einen einzigen Kreisbogen gezeichnet werden sollen.
\selectlanguage{american}

\end{addmargin}
