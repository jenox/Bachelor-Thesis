\chapter{Minimizing Potential Energy}
\label{chap:minimizing-potential-energy}

Potential energy is the capacity for doing work which results only from a system's configuration, \ie{} its particles' positions. Recall that we can express all particle positions ${\vec{r}_i}$ as a function of the generalized coordinates; hence the potential energy ${U}$, too, is a function of the generalized coordinates. In the following, we shall assume the potential energy function to be continuous and locally differentiable.

In conservative systems, mechanical energy is conserved, which is equivalent to the work done by a force being independent of its trajectory. Therefore we can write the forces in terms of the potential energy as
%
\begin{equation}
  \vec{F}_i \coloneqq -\frac{\partial U}{\partial \vec{r}_i}.
  \label{eqn:force-as-differential}
\end{equation}
%
The forces ${\vec{F}_i}$ are defined such that the potential energy ${U}$ is equal to the work one must do \emph{against} the forces to transfer an arbitrary constant reference configuration into the current configuration; hence the negative sign. As a result, the forces are always directed towards a lower energy level and therefore act to reduce the system's potential energy \cite{Nave}.

\hfill

\noindent
A particle is said to be in \emph{mechanical equilibrium} if there is no net force on said particle. Similarly, a particle system is in mechanical equilibrium if the net force on all its particles is zero \cite{Orear}. Using \cref{eqn:force-as-differential}, we find that a system is in equilibrium if and only if the gradient ${\nabla U}$ is zero for its current configuration. For (local) minima of the potential energy, the system is said to be in \emph{stable equilibria}.

Recall that the forces ${\vec{F}_i}$ are defined to act to reduce the system's potential energy. If a system has been displaced from an equilibrium configuration, said forces consequently act to return the system to equilibrium and are therefore called \emph{restoring forces}.

Oftentimes a system has multiple stable equilibria in which algorithms looking to minimize the system's potential energy can get stuck. It is a challenging problem to find a configuration in which the system's energy reaches a global minimum.





\clearpage
\section{Force-Directed Algorithms}
\label{sect:force-directed-algorithms}

As the name suggests, force-directed algorithms define forces between pairs of particles, based on their relative positions, and use these forces to iteratively move the particles, attempting to minimize the system's implicitly defined potential energy \cite{Kobourov}. These forces are defined such that they are restoring forces, \ie{} they point towards equilibrium.

Using the formula for physical work, we can define the potential energy of a configuration as the work one must do against the restoring forces to transfer an arbitrary constant reference configuration into the current configuration.
%
\begin{equation}
  U \coloneqq \int -\vec{F}_\text{res}(\vec{r}) \differential{\vec{r}}
  \label{eqn:energy-as-integral}
\end{equation}
%
For a spring, one could choose the reference configuration such that the spring is relaxed; for two charged particles, one might choose a reference configuration in which they are infinitely far apart. Note that since we are dealing with conservative systems, the concrete trajectory between the endpoints does not matter.

Applying \cref{eqn:energy-as-integral} to each force allows us to calculate the system's total potential energy without ever explicitly defining it. \Cref{eqn:energy-as-integral} also shows that giving in to a restoring force, \ie{} moving a particle by an infinitesimal distance in the direction of the restoring force, decreases the implicitly defined potential energy. Note that there potentially are multiple forces acting on the same particle; therefore one must first calculate the net force acting on each particle as the (vector) sum of the individual restoring forces, and thereby find the direction in which the particles need to be moved in order to reduce the system's potential energy.

When displacing each particle by an infinitesimal distance in the direction of the net force acting on it, it is evident that local energy minima can not be overcome. Depending on the initial configuration, it may not be possible to reach a global energy minimum. Although infinitesimal displacements are not possible in practice, force-directed algorithms with larger displacements generally yield good results \cite{Kobourov}.





\subsection{Generalized Forces}

Forces do not act on (generalized) coordinates; they act on the particles whose positions are determined by the generalized coordinates. Depending on the constraints the system is subject to, a movement in the direction of the restoring forces may or may not be possible. If it is, one must find the adjustment that needs to be made to the generalized coordinates that results in the desired change in particle positions. Considering the positions ${\vec{r}_i}$ are functions of the generalized coordinates ${q \coloneqq (q_1, \ldots, q_m)}$, we can use the restoring forces to find the so-called \emph{generalized forces} acting on the generalized coordinates \cite{Fitzpatrick}:
%
\begin{align}
  Q_j \;\coloneqq &\;\; \sum_{i=1}^{n}{\vec{F}_i \cdot \frac{\partial \vec{r}_i}{\partial q_j}}, \qquad j = 1, \ldots, m
  \label{eqn:generalized-forces-definition} \\
  \;\stackrel{\mathclap{\eqref{eqn:force-as-differential}}}{=} &\;\; -\frac{\partial U}{\partial q_j}
  \label{eqn:generalized-forces-as-gradient}
\end{align}
%
Note that the products ${Q_j \cdot q_j}$ always have the dimension of work. Hence the generalized forces do not necessarily have the dimension of force and instead depend on the dimensions of their corresponding generalized coordinates.



\paragraph{Feasibility}

Both generalized coordinates and generalized forces are scalar quantities. The sign of a generalized force ${Q_j}$ still indicates in which direction its corresponding generalized coordinate ${q_j}$ needs to be adjusted for it to result in the desired change in particle positions, which in turn reduces the system's potential energy. Whether or not it is feasible to use the signs of the generalized forces as a hint to the direction in which to adjust the generalized coordinates very much depends on the complexity of the ${\vec{r}_i(q_1, \ldots, q_m)}$ and their partial derivatives with respect to the generalized coordinates ${q_j}$.

Note that if the system is not subject to any constraints and one uses Cartesian coordinates as the generalized coordinates, the generalized forces essentially become a resolution of the restoring forces into their x and y components, yielding the traditional force-directed algorithms for systems without constraints.



\paragraph{Forces in Equilibrium}

In constrained systems, the constraints implicitly define so-called \emph{constraining forces} which act perpendicularly to the allowed movement to keep all constraints satisfied \cite{Fliessbach}. These forces need to be taken into account for the net force on all particles to be zero in equilibrium configurations.

Back in the example of a mechanical pendulum, the only exterior force is the gravitational force; and it acts on the pendulum regardless of its position. The net force on the weight only becomes zero in equilibrium when taking the implicit constraining force of the rigid rod into account.


\clearpage
\section{Explicit Energy Function}
\label{sect:explicit-energy-function}

Sometimes the restoring forces result in equilibria that do not quite exhibit the desired features, or it is not clear how to choose the restoring forces at all. By instead explicitly assigning each configuration a potential energy, one can easily specify which features are desirable in equilibrium, and which are not. One does not need to provide the restoring forces, \ie{} a direction towards equilibrium \emdash a generic optimization algorithm will figure that out.

Let us assume that all generalized coordinates ${q_j}$ can be (reversibly) transformed to be real-valued. Then we can collect all generalized coordinates in a real-valued vector and write the potential energy function as
%
\begin{equation*}
  U' \colon \mathbb{R}^m \to \mathbb{R} \cup \lbrace \infty \rbrace.
\end{equation*}
%
For invalid configurations, \eg{} if two charged particles coincide, we shall use an infinite potential energy instead of leaving the function undefined. The potential energy function ${U'}$ can then be minimized without any background knowledge of the problem, such as the restoring forces in the system.





\subsection{Derivative-based Optimization}

Many optimization methods require information about partial derivatives of the function to be optimized. These include finding an extremum analytically, but also many numerical optimization techniques such as Newton's method, coordinate descent methods, and conjugate gradient methods. For small systems these methods may be an option; but for larger systems, it is generally infeasible to obtain accurate derivative information.





\subsection{Derivative-free Optimization}

Due to the lack of accurate derivative information, methods that only evaluate function values are often better-suited to minimize the potential energy in larger systems. We shall discuss a simple algorithm in this category in greater detail.



\subsubsection*{Hill Climbing}

Hill climbing is a numeric optimization algorithm that iteratively improves the quality of its solution by adjusting one dimension at a time. In each iteration, the algorithm attempts to adjust a single dimension of its current state and accepts that change if and only if it results in an improvement in value space. This process is repeated until the maximum number of iterations has been performed, or until no further improvements can be found \cite{Russell}. There are three major drawbacks of hill climbing:
%
\begin{enumerate}
  \itemsep 0em
  \item \text{Local Optima:} The algorithm cannot escape a local optimum since adjustments are only accepted if they improve the function evaluation, and may therefore not reach a global optimum.
  \item \text{Ridges and Alleys:} Considering the algorithm adjusts one dimension at a time, the search tends to zig-zag in non-axis-aligned ridges or alleys, taking an unreasonable amount of time to ascend the ridge or descend the alley.
  \item \text{Plateaux:} A plateau is an area in which the value function is essentially flat. Depending on the concrete implementation, the algorithm will either not make any improvements at all, or conduct a random walk.
\end{enumerate}
%
Popular variants of hill climbing include evaluating multiple neighboring states and continuing with the best, and adaptive step sizes for each dimension that change throughout the algorithm. \Cref{algo:adaptive-hill-climbing} shows a possible implementation of adaptive hill climbing.

\hfill

\begin{algorithm}[H]
  \caption{Adaptive Hill Climbing for Minimization}
  \label{algo:adaptive-hill-climbing}
  \SetKwData{Acceleration}{acceleration}
  \SetKwData{Steps}{steps}
  \SetKwData{Factor}{factor}
  \SetArgSty{textrm}
  \vspace{5pt}
  \KwData{number of iterations ${n}$, \newline value function ${f \colon \mathbb{R}^m \to \mathbb{R}}$, \newline start configuration ${\vec{x}_0 \in \mathbb{R}^m}$}
  \KwResult{$\vec{x}_n$ with $f(\vec{x}_n) \leq f(\vec{x}_0)$}
  \vspace{10pt}
  ${\Acceleration \gets 1.25}$\;
  ${\Steps \gets (1, \ldots, 1)}$\;
  \;
  \For{${i \in 1 \ldots n}$}{
    ${\vec{x}_i \gets \vec{x}_{i-1}}$\;
    \;
    \For{${j \in 1 \ldots m}$}{
      ${\vec{x}_{i,j} \gets \vec{x}_{i-1}}$\;
      ${\Factor \gets \Acceleration^{-1}}$\;
      \;
      \For{${k \in -1 \ldots 1}$}{
        ${\vec{x}_{i,+j,k} \gets \vec{x}_{i-1} + e_j \cdot \Steps_j \cdot \Acceleration^k}$\;
        ${\vec{x}_{i,-j,k} \gets \vec{x}_{i-1} - e_j \cdot \Steps_j \cdot \Acceleration^k}$\;
        \;
        \If{${f(\vec{x}_{i,k,+j}) < f(\vec{x}_{i,j})}$} {
          ${\vec{x}_{i,j} \gets \vec{x}_{i,k,+j}}$\;
          ${\Factor \gets \Acceleration^k}$\;
        }
        \;
        \If{${f(\vec{x}_{i,k,-j}) < f(\vec{x}_{i,j})}$} {
          ${\vec{x}_{i,j} \gets \vec{x}_{i,k,-j}}$\;
          ${\Factor \gets \Acceleration^k}$\;
        }
      }
      \;
      ${\Steps_j \gets \Factor \cdot \Steps_j}$\;
      \;
      \If{${f(\vec{x}_{i,j}) < f(\vec{x}_i)}$} {
        ${\vec{x}_i \gets \vec{x}_{i,j}}$\;
      }
    }
  }
  \;
  \Return ${\vec{x}_n}$
\end{algorithm}

\hfill

Considering hill climbing adjusts one dimension at a time, for it to converge to a (local) minimum, it should be started from a valid configuration, \ie{} from one with finite potential energy. Unlike most randomized optimization algorithms, hill climbing has decent intermediate states, allowing for proper visualization of the optimization process.

