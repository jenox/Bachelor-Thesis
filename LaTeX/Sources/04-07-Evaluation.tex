\section{Evaluation}
\label{sect:evaluation}

\newcommand{\expnumber}[2]{{#1}\mathrm{e}{#2}}
\newcommand{\drawing}[1]{\setlength\fboxsep{0pt}\colorbox{gray!10}{\makebox(110,110){\includegraphics[width=115pt,height=115pt,keepaspectratio]{#1}}}}

Besides this written report, we have implemented the algorithms illustrated in \crefrange{sect:existence-of-drawings-with-circular-arcs}{sect:forces-and-potential-energy} in a Mac OS application written in Swift 3. It allows arbitrary input graphs and greedily realizable sequences of paths to be loaded and modified. Unless specified, the initial configuration of the drawing is generated randomly. The user can make manual adjustments to the drawing, but can also use hill climbing to reduce the drawing's energy automatically. Drawings generated by the application can also be exported as vector graphics.

The implementation is open source and can be found on GitHub:
%
\begin{center}
  \url{https://github.com/jenox/bachelor-thesis/}
\end{center}





\hiddensubsection{Results}

We tested the algorithms using random graphs of various orders and densities, which were generated according to the Erdős--Rényi model \cite{Erdos, Gilbert}. The constants in \cref{sect:forces-and-potential-energy} were chosen as ${k = 100}$, ${c_1 = 10^5}$, ${c_2 = 1}$, ${c_3 = 10^4}$, ${c_4 = 10}$, and ${c_5 = 10}$.

\Cref{fig:drawings-by-decomposition-10} shows drawings of graphs with 10 vertices and different densities, each with two different decompositions into greedily realizable sequences of paths: The drawings in the upper row are based on the greedy graph decomposition illustrated in \cref{algo:greedy-graph-decomposition}, whereas the drawings in the lower row were created using user-provided decompositions.

For user-provided graph decompositions, \crefrange{fig:drawings-by-adjustments-10}{fig:drawings-by-adjustments-20} show the effect additional user adjustments have. The drawings in the upper row were created only by hill climbing from a randomly generated start configuration, whereas in the lower row, few user adjustments have been made.

\vspace*{\fill}

\begin{figure}[H]
\centerline{
\drawing{Resources/Figures/Graph-10-09-G.pdf}\hspace{4pt}%
\drawing{Resources/Figures/Graph-10-15-G.pdf}\hspace{4pt}%
\drawing{Resources/Figures/Graph-10-20-G.pdf}\hspace{4pt}%
\drawing{Resources/Figures/Graph-10-25-G.pdf}%
}
\vspace{3pt}
\centerline{
\drawing{Resources/Figures/Graph-10-09-2.pdf}\hspace{4pt}%
\drawing{Resources/Figures/Graph-10-15-2.pdf}\hspace{4pt}%
\drawing{Resources/Figures/Graph-10-20-2.pdf}\hspace{4pt}%
\drawing{Resources/Figures/Graph-10-25-2.pdf}%
}
\caption{Drawings of graphs using decompositions provided by the user (bottom) and greedily determined by \cref{algo:greedy-graph-decomposition} (top).}
\label{fig:drawings-by-decomposition-10}
\end{figure}

\clearpage

\begin{figure}[H]
\centerline{
\drawing{Resources/Figures/Graph-10-09-1.pdf}\hspace{4pt}%
\drawing{Resources/Figures/Graph-10-15-1.pdf}\hspace{4pt}%
\drawing{Resources/Figures/Graph-10-20-1.pdf}\hspace{4pt}%
\drawing{Resources/Figures/Graph-10-25-1.pdf}%
}
\vspace{3pt}
\centerline{
\drawing{Resources/Figures/Graph-10-09-2.pdf}\hspace{4pt}%
\drawing{Resources/Figures/Graph-10-15-2.pdf}\hspace{4pt}%
\drawing{Resources/Figures/Graph-10-20-2.pdf}\hspace{4pt}%
\drawing{Resources/Figures/Graph-10-25-2.pdf}%
}
\caption{Drawings of graphs with 10 vertices and 9/15/20/25 edges; each with (bottom) and without (top) user adjustments.}
\label{fig:drawings-by-adjustments-10}
\end{figure}

\vspace*{1.5cm}

\begin{figure}[H]
\centerline{
\drawing{Resources/Figures/Graph-15-14-1.pdf}\hspace{4pt}%
\drawing{Resources/Figures/Graph-15-20-1.pdf}\hspace{4pt}%
\drawing{Resources/Figures/Graph-15-25-1.pdf}\hspace{4pt}%
\drawing{Resources/Figures/Graph-15-35-1.pdf}%
}
\vspace{3pt}
\centerline{
\drawing{Resources/Figures/Graph-15-14-2.pdf}\hspace{4pt}%
\drawing{Resources/Figures/Graph-15-20-2.pdf}\hspace{4pt}%
\drawing{Resources/Figures/Graph-15-25-2.pdf}\hspace{4pt}%
\drawing{Resources/Figures/Graph-15-35-2.pdf}%
}
\caption{Drawings of graphs with 15 vertices and 14/20/25/35 edges; each with (bottom) and without (top) user adjustments.}
\label{fig:drawings-by-adjustments-15}
\end{figure}

\clearpage

\begin{figure}[H]
\centerline{
\drawing{Resources/Figures/Graph-20-19-1.pdf}\hspace{4pt}%
\drawing{Resources/Figures/Graph-20-30-1.pdf}\hspace{4pt}%
\drawing{Resources/Figures/Graph-20-40-1.pdf}\hspace{4pt}%
}
\vspace{3pt}
\centerline{
\drawing{Resources/Figures/Graph-20-19-2.pdf}\hspace{4pt}%
\drawing{Resources/Figures/Graph-20-30-2.pdf}\hspace{4pt}%
\drawing{Resources/Figures/Graph-20-40-2.pdf}%
}
\caption{Drawings of graphs with 20 vertices and 19/30/40 edges; each with (bottom) and without (top) user adjustments.}
\label{fig:drawings-by-adjustments-20}
\end{figure}





\vspace*{0.5cm}

\hiddensubsection{Conclusion}

The algorithm works nicely with trees and other sparse graphs.

Due to the nature of hill climbing, the algorithm presented here is even more likely to get stuck in local minima than traditional force-directed algorithms. These obstacles are fairly obvious for sparse graphs though, and can easily be overcome using few user adjustments. Even though in theory hill climbing is terribly inefficient, it is highly parallelizable and generally well-suited due to the lack of accurate derivative information of the energy function.

For dense input graphs, on the other hand, the algorithm does not produce satisfying drawings. The hard constraints of vertices on some path ${P}$ lying on the same circular arc ${\Gamma_P}$ take away a lot of freedom in vertex movement that traditional force-directed algorithms have. Possible ways to release tension in the drawing may be prohibited by the constraints, generally yielding drawings with much higher energy values. \Cref{fig:drawings-by-decomposition-10} shows that the decomposition of the input graph ${G}$ into a greedily realizable sequence of simple paths ${\Pi}$ is of the utmost importance here. It is very unlikely that a greedy graph decomposition as illustrated in \cref{algo:greedy-graph-decomposition} yields satisfying results.

The choice of generalized coordinates in \cref{sect:generalized-coordinates} and their transformation in \cref{sect:transformation-of-generalized-coordinates} require the path decomposition to be a greedily realizable sequence of paths. As a result, each vertex can be internal to at most one path, meaning that there are most ${\abs{V}}$ paths that have internal vertices \emdash the rest of the paths, which are ${\bigTheta{\abs{V}^2}}$ many for graphs with ${\bigTheta{\abs{V}^2}}$ edges, are single-edged. Considering the main motivation for using circular arcs to draw multiple incident edges in the first place was drawing graphs with few geometric entities, the algorithms presented here fail for graphs with large dense components by default.
